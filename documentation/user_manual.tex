\documentclass[12pt]{article}

%设置页边距
\usepackage{geometry}
\geometry{left=2.5cm,right=2.5cm,top=2.5cm,bottom=2.5cm}

%地址链接支持
\usepackage{hyperref}

%支持分栏
\usepackage{multicol}

%页眉页脚
\usepackage{fancyhdr}
\pagestyle{fancy}
\lhead{用户手册}
\rhead{User Manual}

%设置字体
\usepackage[no-math]{fontspec}
\usepackage{xunicode}
\usepackage{xltxtra}
\setmainfont[AutoFakeSlant]{Hiragino Sans GB W3}

%设置换行缩进
\usepackage[indentfirst=false,slantfont,boldfont]{xeCJK}
\usepackage{indentfirst}
\setlength{\parindent}{0.9cm}

%行距
\linespread{1.2}

\begin{document}

\title{用户手册:\\[3ex] 基于主页标题文字的\\热点新闻跟踪与分析、可视化展示\\[3ex]}

\author{开发小组}

\date{2014年6月}
\maketitle
\newpage
\renewcommand{\contentsname}{用户手册}
\tableofcontents
\newpage
\section{功能概述}

目前新闻网站内部的新闻有比较好的分类,可以比较容易的收集关于同一个事件的相关新闻。但是收集不同门户网站的相关新闻就显得比较困难。而用户希望得到多个网站相关的新闻信息。所以将门户网站的新闻进行聚类整理就显得十分重要。

同时,如果不同门户网站同时出现了相关新闻,也能反映出当前社会的热点。对门户网站新闻进行聚类也有利于社会热点的研究,如果有大量历史数据的数据库,则在社会研究上也有较好的附加效益。

为此,本项目提供了对于几大新闻网站首页和滚动新闻的标题自动抓取,并能将结果存储至数据库中。需要时,用户可以通过web界面调取一段时间内的新闻标题,并获得聚类后的结果与关键词等信息。

\section{使用方法}

\textbf{注意:程序仅提供了Mac OS X下的二进制程序文件}

打开终端,进入应用程序bin目录,运行focus程序。

程序会加载模型文件,加载完成后会提示用户打开web界面,并在后台定期抓取最新的新闻条目。
用户可以访问http://localhost:8888/查看web界面,在右上角设定开始结束日期后,点击GO按钮获取聚类结果。点击标题可以打开原始新闻网站。对于每一个类别,还可以点击ViewTags按钮查看关键词云。

\end{document}