\documentclass[12pt]{article}

%设置页边距
\usepackage{geometry}
\geometry{left=2.5cm,right=2.5cm,top=2.5cm,bottom=2.5cm}

%地址链接支持
\usepackage{hyperref}

%支持分栏
\usepackage{multicol}

%页眉页脚
\usepackage{fancyhdr}
\pagestyle{fancy}
\lhead{项目进展报告}
\rhead{Project Progress Report}

%设置字体
\usepackage[no-math]{fontspec}
\usepackage{xunicode}
\usepackage{xltxtra}
\setmainfont[AutoFakeSlant]{Hiragino Sans GB W3}

%设置换行缩进
\usepackage[indentfirst=false,slantfont,boldfont]{xeCJK}
\usepackage{indentfirst}
\setlength{\parindent}{0.9cm}

%行距
\linespread{1.2}

\begin{document}

\title{技术报告:\\[3ex] 基于主页标题文字的\\热点新闻跟踪与分析、可视化展示\\[3ex]}

\author{开发小组}

\date{2014年6月}
\maketitle
\newpage
\renewcommand{\contentsname}{技术报告}
\tableofcontents
\newpage
\section{需求分析}
目前新闻网站内部的新闻有比较好的分类,可以比较容易的收集关于同一个事件的相关新闻。但是收集不同门户网站的相关新闻就显得比较困难。但是用户希望得到多个网站相关的新闻信息。所以将门户网站的新闻进行聚类整理就显得十分重要。

同时,如果不同门户网站同时出现了相关新闻,也能反映出当前社会的热点。对门户网站新闻进行聚类也有利于社会热点的研究,如果有大量历史数据的数据库,则在社会研究上也有较好的附加效益。

\section{模块介绍}
实现这些功能需要的步骤如下:
\begin{itemize}
\item 从各种门户网站抓取标题数据
\item 在数据库存储有关数据
\item 用户提交查询的需求
\item 算法对相关标题数据进行聚类
\item 将结果在界面上展示
\end{itemize}

\subsection{网页抓取}

%=======================subsection end===========================

\subsection{数据存储}

%=======================subsection end===========================

\subsection{核心算法}

%=======================subsection end===========================

\subsection{用户交互}

%=======================subsection end===========================

\section{编译方法}

%=======================section    end===========================

\end{document}